\documentclass[sotsuron]{kuee}

\title{チームスポーツの戦術改善に資する選手向け可視化システムの開発}
\author{今井 晨介}
\professor{小山田 耕二 教授}
\course{京都大学工学部}
\department{電気電子工学科 電気工学専攻}
\date{平成27年2月??日}

%%% 本文
\begin{document}
\maketitle	
\tableofcontents

%%%序論
\chapter{序論}
近年、ビッグデータを用いた視覚的分析の需要が高まっており、スポーツの分野でも様々な手法が提案されている。スポーツ分野における既存の視覚的分析補助システムは、試合のデータを用いたデータマイニングを支援し、主に監督、コーチ、または記者をユーザー対象としている。
特に、プロの監督、コーチは試合での戦術戦略に関する意思決定を行う際に視覚的分析補助システムを用いていることがあり、その需要に合ったシステム研究も行われている。[1-4]
また、記者が出来得る限り偏った表現をしないように記事を書く際、スポーツ分析システムを用いて試合情報を読者に伝えられるよう考慮されて開発されたツールも存在する。[4]
\\しかしながら、選手自らが戦術、戦略を練るチームや、選手自らの分析が試合に大きく影響するスポーツの場合、こういった既存のスポーツ分析システムの多くは選手の利用を想定しておらず、操作が複雑、もしくは表現が難解であり、分析に時間を要する事が多い。
事実、選手自らが戦術、戦略を練ることや、選手個人の戦術理解が重要な男子ラクロスというスポーツの場合、選手自らもビデオを観ることで出場した試合を分析し、その試合に用いられた戦術戦略はいかに機能したか、選手たちのコンディションはどうだったのかを振り返り、以後の戦術改善に活かすことが多い。また、これから対戦をする相手チームのビデオを観ることで分析を行うこともある。つまり、この状況下では試合の分析に多くの時間を費やしてしまう。
練習やトレーニングで時間を割くスポーツ選手にとって、分析に要する時間は負担である。
にも関わらず、選手に最適化した、戦術改善し資す分析を可能とする分析補助システムは先行研究は見受けられない。
既存のツールは選手が操作、情報探索するのに不十分な状況である。
\\そこで、本研究では既存システムの複雑さを解消し、選手向けのゲーム可視化ツールを開発した。
まずはじめに、開発に際して本研究では、選手を対象とした分析補助システムへの需要を調査し、そのニーズを確認した。
需要調査には評価グリッド法[]とアンケート手法を用い、需要要項を明らかにし実装を行った。
システム画面を以降はダッシュボードと呼称する。
このダッシュボード上に複数のパネルを配置し、各パネルにそれぞれゲームに関する情報を視覚化したグラフとともに表示し、下記の機能を加えることで選手でも容易に戦術改善に利用可能な有益な情報にたどり着けるシステムを提案する。
	\begin{itemize}
		\item 結果の善し悪しによってパネルの配置、配色を変えることで情報探索補助する機能
		\item パネル間の比較補助機能
		\item インタラクティブに情報探索を行える機能
		\item 特定のプレーを動画にて確認する機能
		\item 試合を俯瞰する機能
	\end{itemize}
ただし、本論文での有益な情報とは、例えば、「どのプレーで負けていたのか、もしくは勝っていたか。どのようなプレーをしていたから勝っていたのか、負けていたのか。誰が要注意人物であるか。」などの選手にアンケート調査を行って要求の高かった項目のことである。
また、上記のシステムの容易性とは、有益な情報と同様にアンケート調査を行った際に要望として高かった容易性に関する項目を指すこととする。
\\本論文の構成は以下のとおりである。
1章,2章,3章,,,


%%%関連研究
\chapter{関連研究}
本章では、スポーツに関する可視化分析システム、及び本提案システムで採用したダッシュボード形式の可視化分析システムについて、本論文での関連研究を述べる。
\\サッカーストーリーという、サッカーの分析ツールを用いることで、サッカーの練習効率を上げ、正しい分析によるコミュニケーションを促すことができると述べている。このシステムではゲームを俯瞰し、また詳細をも見ることができる。記者が記事を書く際にも、試合について監督が分析する時に時間短縮できるようになると主張している。
サッカーストーリーでは一つの試合での出来事をフィールド上にグラフを配置することで、その試合のダイジェストを視覚的に得ることができる。しかしながら、試合間のデータの比較と行った場合には、グラフを並べることが出来ないので不便な部分がある。
\\ーーーらは、これまでに研究されているテニスの可視化技術は、プロ選手の試合放送の質を高め、コーチが選手をコーチングするために、ボールや選手のトラッキングデータを用いることに終始している事に注目した。データの収集、分析はプロでなければコストがかかる。スコアや、ラリーの長さや、サーブの情報や、試合のビデオなどをコンシューマ向けビデオカメラから簡単に試合データを収集し、tennis visシステムにより3種の手法を用いて可視化した。Tennis visシステムを用いることでテニスのコーチ、プレーヤーは試合のパフォーマンスを素早く振り返ることができると主張している。しかしながら、サッカーストーリーと同様、試合間のデータの比較と行った場合には、グラフを並べることが出来ないので不便な部分がある。
\\ーーーらは、イベントの流れを視ることができるoutflowシステムを開発した。Outflowシステムでは特定のイベントフローデータに対しての可視化手法としては優れており、容易にフローの結果を知ることができ、試合結果の原因を探ることができる。
\\しかしながら、上記のスポーツ可視化システムにて、中には選手向けというフレーズも含まれているが、ターゲットを選手としているものの、実装段階、評価段階において選手の意見は含まれていない。
選手自らが戦術、戦略を練るチームや、選手自らの分析が試合に大きく影響するスポーツの場合、こういった既存のスポーツ分析システムの多くは選手の利用を想定しているとはいいつつも、操作が複雑、もしくは表現が難解であり、分析に時間を要する事が多い。
したがって、本論文ではあえてターゲットを選手のみに絞り、選手の目線で必要な要素を考慮して可視化システムを作成している。
また、評価実験に際して、定量的な評価を行わず数名の専門家に対して定性的な意見を得ることに終始しているものが多く、必ずしも有用性が明らかとは言えない。
\\ーーーらは、レイアウト変更可能なダッシュボードフレームワークを作成した。ーーーらの提案しているダッシュボードフレームワークは、レイアウト変更可能なダッシュボード上でのデータ可視化プロセスの自動化をできるよう開発した。このフレームワークではデータの種類には依存せず、また、ユーザー好みの構成にもとづいてグラフを生成することができる。このフレームワークはサードパーティ製やチャートライブラリを用いたグラフの生成が可能であり、柔軟も合わせ持っている。
\\本提案システムでは、既存のダッシュボードフレームワーシステムを参考にしつつ、選手たちの要望に沿ったダッシュボードシステムを作成した。

\chapter{システム要求要件の抽出}
\section{評価グリッド法を用いたニーズの抽出}
ラクロス可視化ツールの要望が何であるかを洗い出し、優先度を決めるためのアンケート質問項目を作成する参考とするために評価グリッド法による予備調査を行った。
\subsection{評価グリッド法}
今回利用する評価グリッド法ついての概要を説明する。
評価グリッド法は、讃井氏が提案した定性評価手法であり、一人ひとりに対してインタビューを行う。
まず評価対象となりうるエレメントを複数提示し、それらエレメントを比較し良い部分を抽出する。
次に、抽出したオリジナル評価項目の上位項目と下位項目を答えてもらい、グラフにマッピングする。
ここで言う上位項目とはオリジナル評価項目を挙げた理由であり、オリジナル評価項目について「〇〇だとなぜいいのですか」と質問を行う。この作業をラダーアップと呼称する。
またこれとは逆に、下位項目はオリジナル評価項目を挙げるにあたり、「具体的にどういうところが〇〇なのか」と質問を行う。この作業をラダーダウンと呼称する。
ラダーアップとラダーダウンを合わせてラダリングと言い、ラダリングされた項目について再度ラダリングを行ない、対象者の回答が引き出せなくなった時点でインタビューを終了する。
これらオリジナル評価項目とラダリングされた項目をマッピングすることで視覚的に認識しやすいグラフが作成される。
以上の手法を取ることにより、評価グリッド法は認知心理学的の立場から人々のある対象に対する評価を明らかにし、深層心理を引き出すことができるとされている。
従って、評価グリッド法は調査対象全体の価値観を把握するのに有効な手段であり、本調査では選手向け可視化ツールの要望を把握する手段として最適と考え、採用した。[][][]
\subsection{調査方法}
京都大学男子ラクロス部に協力して頂き、5名に対して評価グリッドを用いて、効果的なアンケート項目を作成した。
まず、尾上らの開発するEvaluation Grid Methodツールを用いてインタビューを行い、個々人の評価グリッドグラフを作成した。
評価対象として、同部で利用されていたエクセルで表現された表と、ラクロス可視化システムのプロトタイプを提示し、ラダリングを行った。
評価対象比較の際、提示した上記の可視化システムプロトタイプは、ラクロスにおけるある二つのプレーに関して、フィールド図上にプロットを行ったものであった。
次に、この5人のネットワーク図を、項目の重複をマージすることにより、一つのネットワーク図にまとめた。
\subsection{調査結果}
調査で得られたグラフをFigに示す。
\subsection{調査考察}
本調査で得られたネットワーク図からアンケート質問項目を抽出し、インタビューに際して選手たちに共通していた認識についてまとめる。
\\アンケート質問項目に関する考察
\\評価グリッドを行った結果、下位項目には、試合でのどのプレーでの正確性、成功率が良かったのか、悪かったのかといった具体的に選手たちが求める情報が何か明らかになった。
また、上位項目には、「こんな感じであってほしい」といったシステム概要への抽象的な要望が多かった。
アンケート質問項目については、次項にて詳細を記述する。
\\選手の共通認識
\\選手が共通して強く望んでいた一つが分析システムのわかりやすさであった。わかりやすさとは具体的に何を表すかは次項に記す。
やはり彼らスポーツ選手は、練習、トレーニングに加え、ビデオによる戦術分析に多くの時間を割いており、負担を感じていた。戦術分析は必要な要素ではあるので欠かせないが、エクセルでの試合情報の集計では戦術分析に活かせておらず、試合のすべてのビデオを見直すことで戦術分析を行っていると述べていた。そこで、これまでより容易に分析可能な、ラクロスに関するデータ可視化技術を用いた分析補助システムの実用化を望んでいた。
したがって、序論で述べていた、スポーツ現場における本提案システムのニーズを確認した。

\section{アンケートによるシステム実装項目優先度調査}
\subsection{調査方法}
評価グリッド法のみではシステムへの要求項目として数が多く、優先度が明確ではないので、評価グリッド法の結果からアンケート質問項目を作成し、同部に対してアンケート調査を行ない、システム要件を明確にした。
\\システム概要の要望に関する質問項目は評価グリッド法より15項目を抽出した。
選手が分析したい対象に関する質問項目も、評価グリッド法より35項目を抽出した。
また、両項目とも複数選択を可とした。
システムの利用用途、目的に関する質問は自由回答とした。
\\アンケート項目詳細は付録に記載する。
\subsection{調査結果}
同部より78名の回答を得た。アンケート結果を以下に示す。
\begin{itemize}
\item システム概要に関する回答で、選択数の割合が全体回答者数の25\%以上であった項目を挙げる。
	\begin{enumerate}
	\item データを簡単に見ることができる。(43人、69\%)
	\item データの比較をすることができる。(37人、60\%)
	\item イメージしやすい。(32人、52\%)
	\item フィールド図で視覚化されている。(28人、45\%)
	\item グラフを用いている。(23人、37\%)
	\item 一つの画面にデータがまとめられている。(22人、35\%)
	\item 映像を含んでいる。(17人、27\%)
	\end{enumerate}
\item 選手が分析したい対象に関する回答で、選択数の割合が全体回答者数の50\%以上であった項目を挙げる。
	\begin{enumerate}
	\item ショット(46人、74\%)
	\item 1on1(39人、63\%)
	\item グラウンドボール(34人、55\%)
	\item 選手の動き(33人、53\%)
	\item クリア(32人、53\%)
	\item パス(31人、50\%)
	\item 得点(31人、50\%)
	\end{enumerate}
\item システムの利用用途、利用目的に関する質問の回答は、重複が多かったものを挙げる。
	\begin{enumerate}
	\item 対戦相手チームの特徴を把握し、自チームの戦術を練るため。
	\item 試合後に、自チームの反省を行ない、以後の方針を決定するため。
	\item 結果の良かったエレメントと、悪かったエレメントの違いを見出し、新たな視点を得るため。
	\item 試合に出るレギュラーを選定する材料とするため。
	\item データの蓄積を行うことで、最適な戦術を選び出すため。
	\end{enumerate}
\end{itemize}
\subsection{調査考察、実装要件}
評価グリッド法を用いたインタビュー調査とアンケート調査により、具体的なシステム要件を洗い出された。
評価グリッド法を用いることで、精査されたアンケート質問項目が作成され、質問項目に漏れの少ないアンケート調査を行う事により、アンケートのみの調査に比べると、正確な選手のニーズを確認出来たと考えられる。
\begin{itemize}
	\item システム概要に関する回答についての考察
		\begin{description}
			\item [「データを簡単に見ることができる。」:]
			評価グリッド法インタビューで多く聞かれた、分析にかける時間短縮を望んでのことだと考えられる。
			従って要項としては、時間短縮のためにデータへのアクセスビリティを考え、webアプリケーションによる実装を行った。
			同様に下項目と重複するが、視覚的に認識することが可能にすることで、時間短縮を図った。
			また、簡単さという表現から、アプリケーションのわかりやすい操作性も求められていると考えられるので、選手にわかりやすユーザーインターフェイスを実装した。
			\item [「データの比較をすることができる。」:]
			他人、他チーム、他日時とのデータの比較を容易にするべく、本論文で提案するパネルベースのシステム構築を行った。一つのデータのまとまりを一つの「パネル」の中で表現し、複数の「パネル」を画面上に配置し、パネル間の比較を補助する機能を実装した。
			\item [「イメージしやすい。」:]
			可視化技術を用いることにより、ユーザーが生データより認識しやすいグラフを提示する。
			\item [「フィールド図で視覚化されている。」:]
			上の「イメージしやすい」という項目の具体的手段にあたると考えられる。本提案システムでは、「イメージしやすい」グラフを作成するため、積極的にフィールド図を用いて実装を行った。
			\item [「グラフを用いている。」:]
			当項目に関しても、上二項目の具体的手段であると考えられ、本提案システムでは、積極的にグラフを用いてデータの視覚化し、システム実装を行った。
			\item [「一つの画面にデータがまとめられている。」:]
			データ表示ページが複数の画面に分断されているシステムの場合、ページ遷移により理解の妨げになると考えられる。従って、本提案システムでは一つの土台を上に「パネル」を配置する方式を提案する。本論文では、この土台を「ダッシュボード」と呼称する。
			\item [「映像を含んでいる。」:]
			ユーザーが詳細に観たいプレーについては、グラフ上のプロット等を選択することによりインタラクティブに動画を再生できるよう実装した。
		\end{description}
	\item 選手が分析したい対象に関する回答についての考察
		\\回答で得られた7項目より以下の4つのグラフを作成した。詳細については次項で述べる。
			\begin{itemize}
				\item Ground ball
				\item Shots
				\item Score
				\item Ballmove
			\end{itemize}
	\item システム利用用途、利用目的に関する回答についての考察
		\\システム利用用途として様々なエレメントの比較と、試合の俯瞰の要素が求められていると考えられる。
		従って、エレメントの比較と、試合の俯瞰を補助する機能を実装した。
		以上より、システムの特徴的な機能を以下にまとめる。
		\begin{enumerate}	
			\item 試合を俯瞰する機能
			\item 結果の善し悪しによってパネルの配置、配色を変えることで比較分析を補助する機能
			\item インタラクティブに情報探索を行える機能
			\item 特定のプレーを動画にて確認する機能
		\end{enumerate}
\end{itemize}

\chapter{提案システム設計}

\chapter{評価実験}
\section{実験1}
\subsection{評価概要}
\subsection{評価結果}
\subsection{評価考察}
\section{実験3}
\subsection{評価概要}
\subsection{評価結果}
\subsection{評価考察}
\section{実験3}
\subsection{評価概要}
\subsection{評価結果}
\subsection{評価考察}

\chapter{結論}


%======================================================================
%		謝辞
%======================================================================
\begin{acknowledgements}
 本研究を進めるにあたり、有益な御指導、御助言を頂きました京都大学高等教育院の小山田耕二教授、坂本尚久特定助教、学際融合教育研究推進センター政策のための科学ユニットの久木元伸如特定講師に深く感謝致します。
本研究を進めるにあたり、プログラミング技術を始め、様々な御助言を頂きました京都大学工学研究科博士課程2年生の尾上洋介氏に深く感謝致します。
本研究を進めるにあたり、アンケート調査や、システム評価実験に協力して下さった京都大学男子ラクロス部の皆様にはご協力を賜りました。ここに深く御礼申し上げます。
最後に、家族をはじめとする私の学生生活を支えてくださったすべての皆様へ心から感謝の意を表します。
\end{acknowledgements}



%======================================================================
%		参考文献
%======================================================================
\bibliographystyle{kueethesis}
\bibliography{sample}



%======================================================================
%		付録
%======================================================================
\appendix

\end{document}
% Local Variables:
% fill-column: 70
% End:
